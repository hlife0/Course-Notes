\documentclass[10pt, a4paper]{article}
\hbadness=10000
\usepackage{listings}
\usepackage{ulem}
\usepackage{appendix}
\usepackage{amsmath, amsthm, amssymb, amsfonts}
\usepackage{thmtools}
\usepackage{graphicx}
\usepackage{setspace}
\usepackage[top=2.5cm, bottom=3.5cm, left=2.5cm, right=2.5cm]{geometry}
\usepackage{float}
\usepackage{hyperref}
\usepackage[utf8]{inputenc}
\usepackage[english]{babel}
\usepackage{ctex}
\usepackage{framed}
\usepackage[dvipsnames]{xcolor}
\usepackage{tcolorbox}
\usepackage{indentfirst}
\usepackage{titlesec}
\usepackage{sectsty}


\colorlet{LightGray}{White!90!Periwinkle}
\colorlet{LightOrange}{Orange!15}
\colorlet{LightGreen}{Green!15}
\colorlet{LightBlue}{Cyan!15}

\newcommand{\HRule}[1]{\rule{\linewidth}{#1}}

\declaretheoremstyle[name=Example,]{thmsty}
\declaretheorem[style=thmsty,numberwithin=section]{example}
\tcolorboxenvironment{example}{colback=LightBlue, boxrule=0pt}

\declaretheoremstyle[name=Principle,]{thmsty}
\declaretheorem[style=thmsty,numberwithin=section]{principle}
\tcolorboxenvironment{principle}{colback=LightGreen, boxrule=0pt}

\declaretheoremstyle[name=Proposition,]{thmsty}
\declaretheorem[style=thmsty,numberwithin=section]{proposition}
\tcolorboxenvironment{proposition}{colback=LightGray, boxrule=0pt}

\declaretheoremstyle[name=Definition,]{thmsty}
\declaretheorem[style=thmsty,numberwithin=section]{definition}
\tcolorboxenvironment{definition}{colback=LightOrange, boxrule=0pt}


\newenvironment{Solution}{\textbf{Solution.}}

\definecolor{codegreen}{rgb}{0,0.6,0}
\definecolor{codegray}{rgb}{0.5,0.5,0.5}
\definecolor{codepurple}{rgb}{0.58,0,0.82}
\definecolor{backcolour}{rgb}{0.95,0.95,0.92}

\lstdefinestyle{mystyle}{
    backgroundcolor=\color{backcolour},  
    commentstyle=\color{codegreen},
    keywordstyle=\color{magenta},
    numberstyle=\tiny\color{codegray},
    stringstyle=\color{codepurple},
    basicstyle=\ttfamily\footnotesize,
    breakatwhitespace=false,        
    breaklines=true,                
    captionpos=b,                    
    keepspaces=true,                
    numbers=left,                    
    numbersep=5pt,                  
    showspaces=false,                
    showstringspaces=false,
    showtabs=false,                  
    tabsize=2
}
\lstset{style=mystyle}

\setlength{\parindent}{2em}
\subsubsectionfont{\color{Blue}}


% ------------------------------------------------------------------------------
\setstretch{1.0}
% ------------------------------------------------------------------------------
\begin{document}
\title{ \normalsize \textsc{}
\\ [2.0cm]
\HRule{1.5pt} \\ [0.3cm]
\LARGE {\textbf{Matrix Algebra and Applications}
\HRule{1.5pt} \\ [0.6cm]
\LARGE{\textbf{MATH 2111 Lecture Notes}} \vspace*{10\baselineskip}}
}
\date{\today}
\author{\textbf{LI, Hongrui}}  %template borrowed from hlx
\maketitle

\clearpage
\tableofcontents
\newpage


% ------------------------------------------------------------------------------
% Start of Chapter 1
% ------------------------------------------------------------------------------
\section{Systems of Linear Equations}
\newpage


% ------------------------------------------------------------------------------    
% Start of Chapter 2
% ------------------------------------------------------------------------------
\section{Matrix Algebra}
\subsection{Matrix Operations}
\subsubsection*{Sums and Scalar Multiplication}
\begin{example}
    \textbf{Matrix Addition}
    \begin{enumerate}
        \item $A = \begin{bmatrix} 1 & 2 \\ 3 & 4 \end{bmatrix}$, $B = \begin{bmatrix} 5 & 6 \\ 7 & 8 \end{bmatrix}$, then $A + B = \begin{bmatrix} 6 & 8 \\ 10 & 12 \end{bmatrix}$
        \item $A = \begin{bmatrix} 1 & 2 \\ 3 & 4 \end{bmatrix}$, $B = \begin{bmatrix} 5 & 6 \\ 7 & 8 \end{bmatrix}$, then $A - B = \begin{bmatrix} -4 & -4 \\ -4 & -4 \end{bmatrix}$
    \end{enumerate}
\end{example}
Given two matrices $A$ and $B$ of the same size, the sum of $A$ and $B$ is the matrix obtained by adding corresponding elements of $A$ and $B$. The difference of $A$ and $B$ is the matrix obtained by subtracting corresponding elements of $A$ and $B$.
\begin{example}
    \textbf{Scalar Multiplication}
    \begin{enumerate}
        \item $A = \begin{bmatrix} 1 & 2 \\ 3 & 4 \end{bmatrix}$, $k = 2$, then $kA = \begin{bmatrix} 2 & 4 \\ 6 & 8 \end{bmatrix}$
        \item $A = \begin{bmatrix} 1 & 2 \\ 3 & 4 \end{bmatrix}$, $k = -1$, then $kA = \begin{bmatrix} -1 & -2 \\ -3 & -4 \end{bmatrix}$
    \end{enumerate}
\end{example}
Given a matrix $A$ and a scalar $k$, the product of $k$ and $A$ is the matrix whose colums are r times the corresponding columns of $A$.
\begin{proposition}
    \textbf{Properties of Matrix Addition and Scalar Multiplication}
    \begin{enumerate}
        \item $A + B = B + A$
        \item $(A + B) + C = A + (B + C)$
        \item $A + 0 = A$
        \item $A + (-A) = 0$
        \item $k(A + B) = kA + kB$
        \item $(k + l)A = kA + lA$
        \item $k(lA) = (kl)A$
        \item $1A = A$
    \end{enumerate}
\end{proposition}
\textit{Proof.} To prove Prop. 2.1, just consider the definition of matrix addition and scalar multiplication, and apply the Principles of numeral Addition and Multiplication.

\subsubsection*{Matrix Multiplication}
Given two matrices $A$ and $B$, for any vector $\mathbf{x}$, we want to make sure that $(AB)\mathbf{x} = A(B\mathbf{x})$.
Starting from this point of view, we can define the matrix multiplication. \textcolor{red}{Justification of this defination remains to be solved.}
\begin{definition}
    \textbf{Matrix Multiplication}\\
    Let $A$ be an $m\times n$ matrix and $B$ be an $n\times p$ matrix, we now define that $$A B = C = (A\mathbf{b_1}, \cdots, A\mathbf{b_p}).$$
    \textbf{Remark:} The product $AB$ is defined if and only if the number of columns of $A$ is equal to the number of rows of $B$, and the size of the product is $m\times p$.
\end{definition}
\begin{example}
    Find the product of the $AB$, where\[
        A = \begin{pmatrix} -2 & 5 & 0 \\ -1 & 3 & 4 \\ 6 & -8 & -7 \\ -3 & 0 & 9 \end{pmatrix}, B = \begin{pmatrix} 4 & -6 \\ 7 & 1 \\ 3 & 2 \end{pmatrix}.
    \]
    \textbf{Solution:} To avoid making mistaks or losing points, follow strictly this defination and write in steps, or write down the calculation details of each elements.
\end{example}
Note that this defination actually uses the defination of matrix-vector multiplication, which is defined as the linear combination of the columns of the matrix.
In many textbooks, the defination of matrix multiplication is defined in the following way, which is equivalent to the defination above and we regard it as Row-Column Rule for Matrix Multiplication in the context of MATH2111.
In general, we have the following properties of matrix multiplication.
\begin{proposition}
    \textbf{Row-Column Rule for Matrix Multiplication}\\
    Let $A$ be an $m\times n$ matrix and $B$ be an $n\times p$ matrix, we now define that the product $AB$ is the $m\times p$ matrix whose $(i,j)$-entry is the dot product of the $i$-th row of $A$ and the $j$-th column of $B$, that is, $$(AB)_{ij} = \sum_{k=1}^n a_{ik}b_{kj},$$
    where $A=\begin{pmatrix}
            a_{11} & \cdots & a_{1n} \\
            \vdots & \ddots & \vdots \\
            a_{m1} & \cdots & a_{mn}
        \end{pmatrix}, B=\begin{pmatrix}
            b_{11} & \cdots & b_{1p} \\
            \vdots & \ddots & \vdots \\
            b_{n1} & \cdots & b_{np}
        \end{pmatrix}.$
\end{proposition}
\begin{proposition}
    \textbf{Properties of Matrix Multiplication}
    \begin{enumerate}
        \item $A(BC) = (AB)C$
        \item $A(B + C) = AB + AC$
        \item $(A + B)C = AC + BC$
        \item $k(AB) = (kA)B = A(kB)$
        \item $I_mA = A = AI_n$
    \end{enumerate}
    \textcolor{red}{\textbf{Warning:}} \\
    (1)  In general, $AB\neq BA$, that is, the multiplication of matrices is not communtive. Moreover, the product of two matrices may not be defined, that is, $AB$ may not be defined even if $BA$ is defined.\\
    (2)  In general, $AB=AC$ does not imply $B=C$.\\
    (3)  In general, $AB=0$ does not imply $A=0$ or $B=0$.
\end{proposition}

\subsubsection*{Powers and Transpose of a Matrix}
\begin{definition}
    \textbf{Powers of a Matrix}\\
    Let $A$ be a $n\times n$ matrix, then we define the power of $A$ as follows:
    \begin{align*}
        A^0 & = I_n             \\
        A^1 & = A               \\
        A^2 & = A\cdot A        \\
        A^3 & = A\cdot A\cdot A \\
            & \cdots
    \end{align*}
\end{definition}
\begin{definition}
    \textbf{Transpose of a Matrix}\\
    Let $A$ be a $m\times n$ matrix, then the transpose of $A$, denoted by $A^T$, is the $n\times m$ matrix whose $(i,j)$-entry is the $(j,i)$-entry of $A$, that is, $$(A^T)_{ij} = A_{ji}.$$
\end{definition}
\begin{example}
    Find the transpose of the matrix $A = \begin{pmatrix} 1 & 2 & 3 \\ 4 & 5 & 6 \end{pmatrix}$.\\
    \textbf{Solution:} We have $A^T = \begin{pmatrix} 1 & 4 \\ 2 & 5 \\ 3 & 6 \end{pmatrix}$.
\end{example}
\begin{proposition}
    \textbf{Properties of Transpose of a Matrix}
    Let $A$ and $B$ be matrices whose sizes are appropriate for the following operations, and let $k$ be a scalar, then we have
    \begin{enumerate}
        \item $(A^T)^T = A$
        \item $(kA)^T = kA^T$
        \item $(A + B)^T = A^T + B^T$
              \textcolor{Orange}{\item $(AB)^T = B^TA^T$}
        \item $(A^k)^T = (A^T)^k$
    \end{enumerate}
\end{proposition}
\; \; \textit{Proof.} For Prop. 2.4.4, we have $(AB)_{ij} = \sum_{k=1}^n a_{ik}b_{kj}$, then $$(AB)^T_{ij} = (AB)_{ji} = \sum_{k=1}^n a_{jk}b_{ki} = \sum_{k=1}^n b_{ki}a_{jk} = (B^TA^T)_{ij}.$$
Note that in Prop. 2.4.4, $A^T B^T$ may not even be defined if $A$ and $B$ are not square matrices.

\subsection{Inverses of Matrices and Elementary Matrices}
\subsubsection*{Inverses of Matrices}
In many situations, we need to solve the equation $A\mathbf{x} = B$, where $A$ is a matrix and $\mathbf{x}$ is a vector. For numeral calculation, it's natural to find the solution by multiplying both sides by $A^{-1}$ to get $\mathbf{x} = A^{-1}B$. In matrices, in order to solve this equation, we introduce the concept of the inverse of a matrix.
\begin{definition}
    \textbf{Identity Matrices}\\
    The $n\times n$ identity matrix, denoted by $I_n$, is the matrix whose $(i,j)$-entry is 1 if $i=j$ and 0 otherwise, that is, $$(I_n)_{ij} = \begin{cases} 1 & \text{if } i=j \\ 0 & \text{if } i\neq j \end{cases}.$$
\end{definition}
\begin{definition}
    \textbf{Inverses of Matrices}\\
    Let $A$ be a square matrix. If there exists a matrix $B$ such that $$AB = BA = I_n$$, then $B$ is called the inverse of $A$, denoted by $A^{-1}$.\\
    Otherwise, if $A$ does not have an inverse, then $A$ is called singular.
\end{definition}
Note that the inverse of a matrix is unique if it exists, which can be proved by contradiction. Assume that $B$ and $C$ are both inverses of $A$, then we have $$B = BI_n = B(AC) = (BA)C = I_nC = C,$$ which shows that $B=C$. To find the unique inverse of a matrix, we can use the following method.
\begin{example}
    Find the inverse of the matrix $A = \begin{pmatrix} 1 & 2 \\ 3 & 4 \end{pmatrix}$.\\
    \textbf{Solution:} We have $A^{-1} = \frac{1}{ad-bc}\begin{pmatrix} d & -b \\ -c & a \end{pmatrix}$, where $a=1$, $b=2$, $c=3$, $d=4$. Thus, $A^{-1} = \frac{1}{1\cdot 4 - 2\cdot 3}\begin{pmatrix} 4 & -2 \\ -3 & 1 \end{pmatrix} = \begin{pmatrix} -2 & 1 \\ 1.5 & -0.5 \end{pmatrix}$.
\end{example}
\textit{Theorem.} From this example, we can derive a general formula for the inverse of a $2\times 2$ matrix. Let $A = \begin{pmatrix} a & b \\ c & d \end{pmatrix}$, then we have $$A^{-1} = \frac{1}{ad-bc}\begin{pmatrix} d & -b \\ -c & a \end{pmatrix}.$$ 
If $ad-bc=0$, then $A$ is singular and does not have an inverse. In general, we can use the Gauss-Jordan method to find the inverse of a matrix, which will be explained later.
No matter whether a matrix is $2 \times 2$ or not, however, we have the following properties covered in the lectures.
\begin{proposition}
    \textbf{Properties of Inverses of Matrices}
    Let $A$ and $B$ be invertible matrices of the same size, and let $k$ be a scalar, then we have
    \begin{enumerate}
        \item $A^{-1}$ is invertible, and $(A^{-1})^{-1} = A$
        \item $(kA)^{-1} = \frac{1}{k}A^{-1}$
        \item $(A^T)^{-1} = (A^{-1})^T$
        \item $(AB)^{-1} = B^{-1}A^{-1}$
    \end{enumerate}
\end{proposition}
\textit{Proof.} For Prop. 2.5.3, just consider using the Properties of Transpose. For Prop. 2.5.4, we have $$(AB)(B^{-1}A^{-1}) = A(BB^{-1})A^{-1} = AIA^{-1} = AA^{-1} = I.$$
\; \; \;In the end, let's recall the problem of solving the equation $A\mathbf{x} = B$. With the concept of invertible matrices, we can now give a general solution. If $A$ is invertible, then we can multiply both sides by $A^{-1}$ to get $\mathbf{x} = A^{-1}B$. If $A\sigma $ is not invertible, then we can use the Gauss-Jordan method to solve the equation (which will be presented right after this part).
\begin{proposition}
    \textbf{Solution of the Equation $A\mathbf{x} = \mathbf{b}$}\\
    If $A$ is an invertible matrix, then the equation $A\mathbf{x} = \mathbf{b}$ has \textbf{unique} solution $\mathbf{x} = A^{-1} \mathbf{b}$
\end{proposition}
\subsubsection*{Elementry Matrices}
Recall the elementary row operations in Chapter 1, which are the following three operations: (1) Interchange two rows, (2) Multiply a row by a nonzero scalar, (3) Add a multiple of one row to another row. 
If an elementary row operation is performed on the $m\times n$ matrix $A$, the resulting matrix can be written as $EA$, where $E$ is the $m\times m$ elementary matrix created by performing the elementary row operation on the $m\times m$ identity matrix $I_m$.\\
\indent Each elementary matrix $E$ is invertible, and its inverse is also an elementary matrix. The inverse of $E$ is the elementary matrix of the same type that transforms $E$ back to $I_m$.
\begin{proposition}
    An $n\times n$ matrix $A$ is invertible if and only if $A$ is row equivalent to the $n\times n$ identity matrix $I_n$. In this case, any sequence of elementary row operations that reduces $A$ to $I_n$ will also transform $I_n$ into $A^{-1}$.
\end{proposition}
\textit{Proof.} All the following statements are equivalent:
\begin{align*}
    & A \text{ is invertible}                                                                                   \\
    \Leftrightarrow & \text{The equation } A\mathbf{x} = \mathbf{b} \text{ has a unique solution for all vectors } \mathbf{b} \\
    \Leftrightarrow & \text{The equation } A\mathbf{x} = \mathbf{0} \text{ has only the trivial solution}                 \\
    \Leftrightarrow & \text{The reduced row echelon form of } A \text{ is } I_n \; \; (A \sim I_n).                                        \\
\end{align*}
Conversly, if $A\sim I_n$, each row operation corresponds to multiplying A by an elementary matrix, so $E_p \cdots E_2 E_1A=I_n$, where $E_1, E_2, \cdots, E_p$ are elementary matrices. Thus we have $$A^{-1} = E_p \cdots E_2 E_1.$$
\begin{proposition}
    \textbf{An algorithm for finding the inverse of a matrix (Gauss-Jordan method)}\\
    Row operation reduce the matrix $\begin{pmatrix} A & I_n \end{pmatrix}$. If $A$ is row equivalent to $I_n$, then $\begin{pmatrix} A & I_n \end{pmatrix}$ is row equivalent to $\begin{pmatrix} I_n & A^{-1} \end{pmatrix}$. Otherwise, $A$ is not invertible.
\end{proposition}
\begin{example}
    Find the inverse of the matrix $A = \begin{pmatrix} 1 & 2 \\ 3 & 4 \end{pmatrix}$.\\
    \textbf{Solution:} We can use the following method to find the inverse of a matrix.\\
    \textbf{Step 1:} Write down the augmented matrix $\begin{pmatrix} A & I_2 \end{pmatrix} = \begin{pmatrix} 1 & 2 & 1 & 0 \\ 3 & 4 & 0 & 1 \end{pmatrix}$.\\
    \textbf{Step 2:} Use the row operations to transform the left side of the augmented matrix into the identity matrix: 
    \(\begin{pmatrix} 1 & 2 & 1 & 0 \\ 3 & 4 & 0 & 1 \end{pmatrix} \rightarrow \begin{pmatrix} 1 & 2 & 1 & 0 \\ 0 & -2 & -3 & 1 \end{pmatrix} \rightarrow \begin{pmatrix} 1 & 2 & 1 & 0 \\ 0 & 1 & 3/2 & -1/2 \end{pmatrix} \rightarrow \begin{pmatrix} 1 & 0 & -1 & 1 \\ 0 & 1 & 3/2 & -1/2 \end{pmatrix}.\)\\
    \textbf{Step 3:} The right side of the augmented matrix is the inverse of the matrix $A$.
\end{example}
\indent \textit{Proof.} $E_p \cdots E_1\begin{pmatrix} A & I_n \end{pmatrix} = \begin{pmatrix}E_p \cdots E_1 A & E_p \cdots E_1 I_n \end{pmatrix} = \begin{pmatrix} I_n & A^{-1} \end{pmatrix}$ 

\subsection{Characterization of Invertible Matrices}
\begin{proposition}
    Let $A$ be an $n\times n$ matrix. The following statements are equivalent:
    \begin{enumerate}
        \item $A$ is invertible.
        \item $A$ is row equivalent to $I_n$.
        \item $A$ has $n$ pivot positions.
        \item The equation $A\mathbf{x} = \mathbf{0}$ has only the trivial solution.
        \item The equation $A\mathbf{x} = \mathbf{b}$ has a unique solution for all vectors $\mathbf{b}$.
        \item The columns of $A$ form a linearly independent set.
        \item The columns of $A$ span $\mathbb{R}^n$.
        \item The linear transformation $\mathbf{x} \mapsto A\mathbf{x}$ is one-to-one.
        \item The linear transformation $\mathbf{x} \mapsto A\mathbf{x}$ is onto.
        \item There exists an $n\times n$ matrix $C$ such that $CA = I_n$.
        \item There exists an $n\times n$ matrix $D$ such that $AD = I_n$.
        \item $A^T$ is invertible.
    \end{enumerate}
\end{proposition}
\textit{Proof.} The equivalence of the statements can be proved by the following chain of implications:
\begin{align*}
    & (1) \Rightarrow (2) \Rightarrow (3) \Rightarrow (4) \Rightarrow (5) \Rightarrow (6) \Rightarrow (7) \Rightarrow (8) \Rightarrow (9) \Rightarrow (10) \Rightarrow (11) \Rightarrow (12) \Rightarrow (1) \\
    & (1) \Rightarrow (12) \Rightarrow (11) \Rightarrow (10) \Rightarrow (9) \Rightarrow (8) \Rightarrow (7) \Rightarrow (6) \Rightarrow (5) \Rightarrow (4) \Rightarrow (3) \Rightarrow (2) \Rightarrow (1)
\end{align*}
\subsubsection*{Invertible Linear transformations}
A linear transformation $T: \mathbb{R}^n \rightarrow \mathbb{R}^n$ is invertible if and only if there exists a map $S: \mathbb{R}^n \rightarrow \mathbb{R}^n$ such that $T\circ S (\mathbf{x}) = S\circ T (\mathbf{x}) = \mathbf{x}$ for all $\mathbf{x} \in \mathbb{R}^n$. In this case, $S$ is called the inverse of $T$, denoted by $T^{-1}$.\\
\indent Let $A$ be the standard matrix of the linear transformation $T$, then $T$ is invertible if and only if $A$ is invertible. In this case, the standard matrix of $T^{-1}$ is $A^{-1}$.\\
\indent To bring this chapter to an end, we show the following proposition given by \textit{Github Copilot}.
\begin{proposition}
    Let $T: \mathbb{R}^n \rightarrow \mathbb{R}^n$ be a linear transformation. The following statements are equivalent:
    \begin{enumerate}
        \item $T$ is invertible.
        \item $T$ is one-to-one.
        \item $T$ is onto.
    \end{enumerate}
\end{proposition}



\newpage
\section{Determinants}
\subsection{Introduction to Determinants}
\begin{definition}
    \textbf{Determinant of a Matrix}\\
    Let $A$ be a $2\times 2$ matrix, then the determinant of $A$, denoted by $\det(A)$ or $|A|$, is defined as the following $$\det(A) = \begin{vmatrix} a & b \\ c & d \end{vmatrix} = ad - bc$$
    In general, the determinant of an $n\times n$ matrix $A$, and $A_{ij}$ is the $(n-1)\times (n-1)$ matrix obtained by deleting the $i$-th row and $j$-th column of $A$, is defined as $\det(A) = \sum_{j=1}^n (-1)^{i+j}a_{ij}\det(A_{ij}).$
    \textbf{Cofactor}\\
    The $(i,j)$-cofactor of $A$ is defined as $C_{ij} = (-1)^{i+j}\det(A_{ij})$.\\
\end{definition}
The defination of determinant is a recursive defination, which is based on the defination of determinant of $2\times 2$ matrix. The determinant of a matrix can be calculated by the following formula:
\begin{proposition}
    \textbf{Calculation of Determinant}\\
    Let $A$ be an $n\times n$ matrix, then the determinant of $A$ can be computed by a cofactor expansion across any row or down any column, that is, 
    \begin{align*}
        \det(A) & = \sum_{j=1}^n a_{ij}C_{ij} \text{ (expansion across the $i$-th row)} \\
                & = \sum_{i=1}^n a_{ij}C_{ij} \text{ (expansion down the $j$-th column)}.
    \end{align*}
    \textbf{Thm for triangular matrix}\\
    If $A$ is a triangular matrix, then $\det(A)$ is the product of the main diagonal entries of $A$.\[
        \begin{pmatrix}
            * & * & * &  \cdots & * \\
            0 & * & * & \cdots & * \\
            0 & 0 & * & \cdots & * \\
            \vdots & \vdots & \vdots & \ddots & \vdots \\
            0 & 0 & 0 & \cdots & *
        \end{pmatrix}
        \text{ or }
        \begin{pmatrix}
            * & 0 & 0 &  \cdots & 0 \\
            * & * & 0 & \cdots & 0 \\
            * & * & * & \cdots & 0 \\
            \vdots & \vdots & \vdots & \ddots & \vdots \\
            * & * & * & \cdots & *
        \end{pmatrix}
    \]
    \color{red}{\textbf{Warning:}} Mind the sign of the cofactor expansion - indexes changed after each recursive call !!
\end{proposition}

\subsection{Properties of Determinants}
\begin{proposition}
    \textbf{Row Operations}\\
    Let $A$ be an $n\times n$ matrix.
    \begin{itemize}
        \item If a multiple of one row of $A$ is added to another row to produce a matrix $B$, then $\det(B) = \det(A)$.
        \item If two rows of $A$ are interchanged to produce a matrix $B$, then $\det(B) = -\det(A)$.
        \item If one row of $A$ is multiplied by $k$ to produce a matrix $B$, then $\det(B) = k\det(A)$.
    \end{itemize}
\end{proposition}
    

\end{document}