\documentclass[10pt, a4paper]{article}
\hbadness=10000
\usepackage{listings}
\usepackage{ulem}
\usepackage{appendix}
\usepackage{amsmath, amsthm, amssymb, amsfonts}
\usepackage{thmtools}
\usepackage{graphicx}
\usepackage{setspace}
\usepackage[top=2.5cm, bottom=3.5cm, left=2.5cm, right=2.5cm]{geometry}
\usepackage{float}
\usepackage{hyperref}
\usepackage[utf8]{inputenc}
\usepackage[english]{babel}
\usepackage{ctex}
\usepackage{framed}
\usepackage[dvipsnames]{xcolor}
\usepackage{tcolorbox}
\usepackage{indentfirst}
\usepackage{titlesec}
\usepackage{sectsty}

\newcommand{\R}{\mathbb{R}}


\colorlet{LightGray}{White!90!Periwinkle}
\colorlet{LightOrange}{Orange!15}
\colorlet{LightGreen}{Green!15}
\colorlet{LightBlue}{Cyan!15}

\newcommand{\HRule}[1]{\rule{\linewidth}{#1}}

\declaretheoremstyle[name=Example,]{thmsty}
\declaretheorem[style=thmsty,numberwithin=section]{example}
\tcolorboxenvironment{example}{colback=LightBlue, boxrule=0pt}

\declaretheoremstyle[name=Principle,]{thmsty}
\declaretheorem[style=thmsty,numberwithin=section]{principle}
\tcolorboxenvironment{principle}{colback=LightGreen, boxrule=0pt}

\declaretheoremstyle[name=Proposition,]{thmsty}
\declaretheorem[style=thmsty,numberwithin=section]{proposition}
\tcolorboxenvironment{proposition}{colback=LightGray, boxrule=0pt}

\declaretheoremstyle[name=Definition,]{thmsty}
\declaretheorem[style=thmsty,numberwithin=section]{definition}
\tcolorboxenvironment{definition}{colback=LightOrange, boxrule=0pt}


\newenvironment{Solution}{\textbf{Solution.}}

\definecolor{codegreen}{rgb}{0,0.6,0}
\definecolor{codegray}{rgb}{0.5,0.5,0.5}
\definecolor{codepurple}{rgb}{0.58,0,0.82}
\definecolor{backcolour}{rgb}{0.95,0.95,0.92}

\lstdefinestyle{mystyle}{
    backgroundcolor=\color{backcolour},  
    commentstyle=\color{codegreen},
    keywordstyle=\color{magenta},
    numberstyle=\tiny\color{codegray},
    stringstyle=\color{codepurple},
    basicstyle=\ttfamily\footnotesize,
    breakatwhitespace=false,        
    breaklines=true,                
    captionpos=b,                    
    keepspaces=true,                
    numbers=left,                    
    numbersep=5pt,                  
    showspaces=false,                
    showstringspaces=false,
    showtabs=false,                  
    tabsize=2
}
\lstset{style=mystyle}

\setlength{\parindent}{2em}
\subsubsectionfont{\color{Blue}}


% ------------------------------------------------------------------------------
\setstretch{1.0}
% ------------------------------------------------------------------------------
\begin{document}
\title{ \normalsize \textsc{}
\\ [2.0cm]
\HRule{1.5pt} \\ [0.3cm]
\LARGE {\textbf{Matrix Algebra and Applications}
\HRule{1.5pt} \\ [0.6cm]
\LARGE{\textbf{MATH 2111 Lecture Notes}} \vspace*{10\baselineskip}}
}
\date{\today}
\author{\textbf{LI, Hongrui}}  %template borrowed from hlx
\maketitle

\clearpage
\tableofcontents
\newpage


% ------------------------------------------------------------------------------
% Start of Chapter 1
% ------------------------------------------------------------------------------
\section{Systems of Linear Equations}
\section{Matrix Algebra}
\newpage



% ------------------------------------------------------------------------------
% Start of Chapter 3
% ------------------------------------------------------------------------------
\section{Determinants}
\subsection{Introduction to Determinants}
At the end of Chapter 2, we have discussed the invertibility of a linear transfermation $T:\R^T\mapsto\R^T$, whose standard matrix happens to be a square matrix. In this chapter, we will introduce the concept of determinant, which is a scalar value that can be computed from the elements of a square matrix and encodes certain properties of the linear transformation described by the matrix. The determinant of a matrix is a fundamental concept in linear algebra, and it is used in many other areas of mathematics.\\
Before we introduce the definition and properties of determinants, we claim that the thing called "determinant of a square matrix" has the following property:
\begin{proposition}
    \textbf{Invertibility and Determinants} \\
    Let $A$ be an $n\times n$ matrix. Then $A$ is invertible if and only if $\det(A)\neq 0$.\\
    \textbf{Corollary} \\
    $\operatorname{det}A=0$ if and only if rows of $A$ are linearly dependent.
\end{proposition}
The next step is to give a defination of determinants. It's easy to observe that linear dependence is obvious when two  columns or two rows are the same, or a column or a row is zero.
Starting from a $2\times 2$ matrix, we find the following defination of determinant will work perfectly for what we want to be, and then we extend it to $n\times n$ matrix by recursion. \\
\indent\textcolor{red}{Remark:} The handwritten lecture notes presents the proposition above explicitly after defining the determinant of a matrix and showing it's properties. However, I think it's better to present the proposition first, since it gives a clear picture of why we need to introduce the concept of determinant. \\
\begin{definition}
    \textbf{Determinant of a Matrix}\\
    Let $A$ be a $2\times 2$ matrix, then the determinant of $A$, denoted by $\det(A)$ or $|A|$, is defined as the following $$\det(A) = \begin{vmatrix} a & b \\ c & d \end{vmatrix} = ad - bc$$
    In general, the determinant of an $n\times n$ matrix $A$, and $A_{ij}$ is the $(n-1)\times (n-1)$ matrix obtained by deleting the $i$-th row and $j$-th column of $A$, is defined as $\det(A) = \sum_{j=1}^n (-1)^{i+j}a_{ij}\det(A_{ij}).$
    \textbf{Cofactor}\\
    The $(i,j)$-cofactor of $A$ is defined as $C_{ij} = (-1)^{i+j}\det(A_{ij})$.\\
\end{definition}
\indent\textit{Proof.} Given this defination, we can easily verify Prop. 3.1. Suppose $A$ has been reduced to an echelon form $U$ by row replacement and row interchanges. If there are r interchanges, then $$
det A = (-1)^r det U = (-1)^r \prod_{i=1}^n u_{ii}.
$$
The defination of determinant is a recursive defination, which is based on the defination of determinant of $2\times 2$ matrix. The determinant of a matrix can be calculated by the following formula:
\begin{proposition}
    \textbf{Calculation of Determinant}\\
    Let $A$ be an $n\times n$ matrix, then the determinant of $A$ can be computed by a cofactor expansion across any row or down any column, that is, 
    \begin{align*}
        \det(A) & = \sum_{j=1}^n a_{ij}C_{ij} \text{ (expansion across the $i$-th row)} \\
                & = \sum_{i=1}^n a_{ij}C_{ij} \text{ (expansion down the $j$-th column)}.
    \end{align*}
    \textbf{Thm for triangular matrix}\\
    If $A$ is a triangular matrix, then $\det(A)$ is the product of the main diagonal entries of $A$.\[
        \begin{pmatrix}
            * & * & * &  \cdots & * \\
            0 & * & * & \cdots & * \\
            0 & 0 & * & \cdots & * \\
            \vdots & \vdots & \vdots & \ddots & \vdots \\
            0 & 0 & 0 & \cdots & *
        \end{pmatrix}
        \text{ or }
        \begin{pmatrix}
            * & 0 & 0 &  \cdots & 0 \\
            * & * & 0 & \cdots & 0 \\
            * & * & * & \cdots & 0 \\
            \vdots & \vdots & \vdots & \ddots & \vdots \\
            * & * & * & \cdots & *
        \end{pmatrix}
    \]
    \color{red}{\textbf{Warning:}} 
    \begin{itemize}
        \item Mind the sign of each new determinant - indexes changed after each recursive call !!
        \item Remember to multiply the coefficient of each $a_{ij}$ in the final results !!
    \end{itemize}
    
\end{proposition}

\subsection{Properties of Determinants}
\subsubsection*{Row and Column Operations}
\begin{proposition}
    \textbf{Row Operations}\\
    Let $A$ be an $n\times n$ matrix.
    \begin{itemize}
        \item If a multiple of one row of $A$ is added to another row to produce a matrix $B$, then $\det(B) = \det(A)$.
        \item If two rows of $A$ are interchanged to produce a matrix $B$, then $\det(B) = -\det(A)$.
        \item If one row of $A$ is multiplied by $k$ to produce a matrix $B$, then $\det(B) = k\det(A)$.
    \end{itemize}
\end{proposition}
\begin{example}
    \textbf{Example} \\
    Let $A = \begin{pmatrix} 
        2 & -8 & 6 & 8\\
        3 &-9 & 5 & 10\\
        -3 & 0 & 1 & -2\\
        1 & -4 & 0 & 6
     \end{pmatrix}$. Calculate $\det(A)$.\\
    \textbf{Solution} 
    The answer is $-36$ (please take care of all the calculation details: every step of the row operation my cause mistakes).
\end{example}
Now that we have the theorems for row operations, it's natural to think about whether the same properties works for row operations. The answer is yes, and the justification just follows the following theorem:
\begin{quote}
    If $A$ is an $n\times n$ matrix, then $\det(A^T) = \det(A)$.
\end{quote}
\subsubsection*{Determinants of Products}
\begin{proposition}
    \textbf{Determinants of Products}\\
    Let $A$ and $B$ be $n\times n$ matrices. Then $\det(AB) = \det(A)\det(B)$.\\
    \textcolor{red}{Remark:} det($A+B$) is not det($A$) + det($B$).
\end{proposition}
\indent Using this proposition, it's easy to show that if $A$ is invertible, then $$\det(A^{-1}) = \frac{1}{\det(A)}$$
\subsubsection*{A Linearity Property of Determinant Function}
We have the following theorem:
\begin{align*}
    &\det(\mathbf{a_1},\cdots,\mathbf{a_{j-1}},\mathbf{a_j}+\mathbf{b_j},\mathbf{a_{j+1}},\cdots,\mathbf{a_n})\\
     = &\det(\mathbf{a_1},\cdots,\mathbf{a_{j-1}},\mathbf{a_j},\mathbf{a_{j+1}},\cdots,\mathbf{a_n}) + \det(\mathbf{a_1},\cdots,\mathbf{a_{j-1}},\mathbf{b_j},\mathbf{a_{j+1}},\cdots,\mathbf{a_n}).
\end{align*}

\subsection{Cramer's Rule, Volume, and Linear Transformations}
\subsubsection*{Cramer's Rule}
\indent To present Cramer's Rule, we first introduce a notation: For any $n\times n$ matrix $A$ and any $\mathbf{b}$ in $\R^n$, let $A_i(\mathbf{b})$ be the matrix obtained by replacing the $i$-th column of $A$ by the column vector $b$. In other words, 
\begin{align*}
    A = (a_1,\cdots,a_i,\cdots,a_n) \text{ and } A_i(\mathbf{b}) = (a_1,\cdots,\mathbf{b},\cdots,a_n).
\end{align*}
\begin{proposition}
    \textbf{Cramer's Rule}\\
    Let $A$ be an $n\times n$ matrix and $\mathbf{b}$ be in $\R^n$. If $\det(A)\neq 0$, then the unique solution of $A\mathbf{x} = \mathbf{b}$ is given by 
    \begin{align*}
        x_i = \frac{\det(A_i(\mathbf{b}))}{\det(A)} \text{ for } i = 1,2,\cdots,n.
    \end{align*}
\end{proposition}
\indent \textit{Proof.} Since the determinant of an arbitary matrix $A$ is quite a complicated problem, we try to get as many "0" as possible in the matrix when calculating the determinant. Hence, we can make use of identity matrix $I_n = (\mathbf{e_1}, \cdots, \mathbf{e_i}, \cdots, \mathbf{e_n})$ and $I_n(\mathbf{b}) = (\mathbf{e_1}, \cdots, \mathbf{b}, \cdots, \mathbf{e_n})$. Then we have
\begin{align*}
    A\cdot I_n(\mathbf{b}) &= (A\mathbf{e_1}, \cdots, A\mathbf{b}, \cdots, A\mathbf{e_n}) \\
    &= (\mathbf{a_1}, \cdots, \mathbf{b}, \cdots, \mathbf{a_n}) \\
    &= A_i(\mathbf{b}).\\
    det(A\cdot I_n(\mathbf{b})) &= det(A)\cdot det(I_n(\mathbf{b})) 
\end{align*}
When calculating the of $I_n(\mathbf{b})$, simply look at the $i$-th row of $I_n$, which only have one non-zero entry $x_i$, then we have $$
det(I_n(\mathbf{b})) = (-1)^{i+i} \cdot x_i \cdot det(I_{n-1}) = x_i.
$$
Hence, we have $x_i = \frac{det(A_i(\mathbf{b}))}{det(A)}$.\\
\indent There are many exercises of solving linear systems in previous chapters, we don't give more examples here. The readers can solve the previous examples using the Cramer's Rule to check the correctness of the results.

\subsubsection*{A formula for $A^{-1}$}
\begin{proposition}
    Let $A$ be an $n\times n$ matrix. If $\det(A)\neq 0$, then $A^{-1} = \frac{1}{\det(A)}\text{adj}(A)$, where $\text{adj}(A)$ is the adjugate matrix of $A$:
    \begin{align*}
        \text{adj}(A) = \begin{pmatrix}
            C_{11} & C_{21} & \cdots & C_{n1} \\
            C_{12} & C_{22} & \cdots & C_{n2} \\
            \vdots & \vdots & \ddots & \vdots \\
            C_{1n} & C_{2n} & \cdots & C_{nn}
        \end{pmatrix}
    \end{align*}, where $C_{ij}$ is the $(i,j)$-cofactor of $A$, i.e. $C_{ij} = (-1)^{i+j}\det(A_{ij})$. 
\end{proposition}
\indent\textcolor{red}{Warning} DO NOT mistake the meaning of $C_{ij}$, which is $(-1)^{i+j}$ times the determinant of the $(n-1)\times (n-1)$ matrix obtained by deleting the $i$-th row and $j$-th column of $A$. (NOT simply $(-1)^{i+j}a_{ij}$)\\
\indent For an example of this rule, please refer to the end of FILE Lecture 10.

\end{document}