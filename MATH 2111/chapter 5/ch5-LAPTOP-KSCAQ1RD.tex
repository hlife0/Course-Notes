\documentclass[10pt, a4paper]{article}
\hbadness=10000
\usepackage{listings}
\usepackage{ulem}
\usepackage{appendix}
\usepackage{amsmath, amsthm, amssymb, amsfonts}
\usepackage{thmtools}
\usepackage{graphicx}
\usepackage{setspace}
\usepackage[top=2.5cm, bottom=3.5cm, left=2.5cm, right=2.5cm]{geometry}
\usepackage{float}
\usepackage{hyperref}
\usepackage[utf8]{inputenc}
\usepackage[english]{babel}
\usepackage{ctex}
\usepackage{framed}
\usepackage[dvipsnames]{xcolor}
\usepackage{tcolorbox}
\usepackage{indentfirst}
\usepackage{titlesec}
\usepackage{sectsty}

\newcommand{\R}{\mathbb{R}}
\newcommand{\vt}[1]{\mathbf{#1}}


\colorlet{LightGray}{White!90!Periwinkle}
\colorlet{LightOrange}{Orange!15}
\colorlet{LightGreen}{Green!15}
\colorlet{LightBlue}{Cyan!15}

\newcommand{\HRule}[1]{\rule{\linewidth}{#1}}

\declaretheoremstyle[name=Example,]{thmsty}
\declaretheorem[style=thmsty,numberwithin=section]{example}
\tcolorboxenvironment{example}{colback=LightBlue, boxrule=0pt}

\declaretheoremstyle[name=Principle,]{thmsty}
\declaretheorem[style=thmsty,numberwithin=section]{principle}
\tcolorboxenvironment{principle}{colback=LightGreen, boxrule=0pt}

\declaretheoremstyle[name=Proposition,]{thmsty}
\declaretheorem[style=thmsty,numberwithin=section]{proposition}
\tcolorboxenvironment{proposition}{colback=LightGray, boxrule=0pt}

\declaretheoremstyle[name=Definition,]{thmsty}
\declaretheorem[style=thmsty,numberwithin=section]{definition}
\tcolorboxenvironment{definition}{colback=LightOrange, boxrule=0pt}


\newenvironment{Solution}{\textbf{Solution.}}

\definecolor{codegreen}{rgb}{0,0.6,0}
\definecolor{codegray}{rgb}{0.5,0.5,0.5}
\definecolor{codepurple}{rgb}{0.58,0,0.82}
\definecolor{backcolour}{rgb}{0.95,0.95,0.92}

\lstdefinestyle{mystyle}{
    backgroundcolor=\color{backcolour},  
    commentstyle=\color{codegreen},
    keywordstyle=\color{magenta},
    numberstyle=\tiny\color{codegray},
    stringstyle=\color{codepurple},
    basicstyle=\ttfamily\footnotesize,
    breakatwhitespace=false,        
    breaklines=true,                
    captionpos=b,                    
    keepspaces=true,                
    numbers=left,                    
    numbersep=5pt,                  
    showspaces=false,                
    showstringspaces=false,
    showtabs=false,                  
    tabsize=2
}
\lstset{style=mystyle}

\setlength{\parindent}{2em}
\subsubsectionfont{\color{Blue}}


% ------------------------------------------------------------------------------
\setstretch{1.0}
% ------------------------------------------------------------------------------
\begin{document}
\title{ \normalsize \textsc{}
\\ [2.0cm]
\HRule{1.5pt} \\ [0.3cm]
\LARGE {\textbf{Matrix Algebra and Applications}
\HRule{1.5pt} \\ [0.6cm]
\LARGE{\textbf{MATH 2111 Lecture Notes}} \vspace*{10\baselineskip}}
}
\date{\today}
\author{\textbf{LI, Hongrui}}  %template borrowed from hlx
\maketitle

\clearpage
\tableofcontents
\newpage


% ------------------------------------------------------------------------------
% Start of Chapter 1
% ------------------------------------------------------------------------------
\section{Systems of Linear Equations}
\section{Matrix Algebra}
\section{Determinants}
\section{Vector Spaces}
\newpage



% ------------------------------------------------------------------------------
% Start of Chapter 4
% ------------------------------------------------------------------------------
\section{Eigenvalues and Eigenvectors}
\subsection{Eigenvalues and Eigenvectors}
\begin{definition}
    Let $A$ be an $n \times n$ matrix. A scalar $\lambda$ is called an \textbf{eigenvalue} of $A$ if there exists a \textit{nonzero} vector $\vt{x}$ in $\R^n$ such that
    \begin{equation}
        A\vt{x} = \lambda \vt{x}
    \end{equation}
    The vector $\vt{x}$ is called an \textbf{eigenvector} of $A$ corresponding to $\lambda$.
\end{definition}
\begin{example}
    Show that 7 is an eigenvalue of the matrix
    \begin{equation}
        A = \begin{bmatrix}
            1&6\\
            5&2
        \end{bmatrix}
    \end{equation}
    and find the corresponding eigenvectors.
    \textbf{Solution.} The scalar 7 is an eigenvalue of $A$ if and only if the equation\[A\vt{x}=7\vt{x}=7I_2\vt{x}\] has a nontrivial solution, which is equivalent to 
    \begin{equation}
        (A-7I_2)\vt{x} = \vt{0}.
    \end{equation}
    We perform the row reduction on the augmented matrix (details should be shown in the exam), and get the result that $x_1-x_2=0$. Hence, the general solution is 
    \begin{equation}
        \vt{x} = t\begin{bmatrix}
            1\\
            1
        \end{bmatrix}
    \end{equation}
    So $(1,1)^T$ is an eigenvector of $A$ corresponding to the eigenvalue 7.
\end{example}
Instead of giving the general solution, we first consider the special case when the matrix is triangular.
\begin{proposition}
    Let $A$ be an $n \times n$ upper triangular matrix. Then the eigenvalues of $A$ are the diagonal entries of $A$.
\end{proposition}
\textit{Proof.} Let $A$ be an $n \times n$ upper triangular matrix. Then we have \[
A=\begin{bmatrix}
    a_{11}&a_{12}&\cdots&a_{1n}\\
    0&a_{22}&\cdots&a_{2n}\\
    \vdots&\vdots&\ddots&\vdots\\
    0&0&\cdots&a_{nn}
\end{bmatrix}
, A-\lambda I_n = \begin{bmatrix}
    a_{11}-\lambda&a_{12}&\cdots&a_{1n}\\
    0&a_{22}-\lambda&\cdots&a_{2n}\\
    \vdots&\vdots&\ddots&\vdots\\
    0&0&\cdots&a_{nn}-\lambda
\end{bmatrix}
\]
Hence we have the following statements equivalent:
\begin{itemize}
    \item $\lambda$ is an eigenvalue of $A$.
    \item The system $(A-\lambda I_n)\vt{x}=\vt{0}$ has a nontrivial solution.
    \item The system $(A-\lambda I_n)\vt{x}=\vt{0}$ has a free variable.
    \item At least one of the entries on the diagonal of $A-\lambda I_n$ is zero.
    \item $\lambda = a_{ii}$ for some $i$.
\end{itemize}
Remark: For a lower triangular matrix, the proof is similar and we have the same conclusion. 0 can also be an eigenvalue.

\begin{proposition}
    If $\vt{v_1}, \vt{v_2}, \cdots, \vt{v_k}$ are eigenvectors of a matrix $A$ corresponding to distinct eigenvalues $\lambda_1, \lambda_2, \cdots, \lambda_k$, then the set $\{\vt{v_1}, \vt{v_2}, \cdots, \vt{v_k}\}$ is linearly independent.
\end{proposition}
\textit{Proof.} [This proof is given by \textit{Github Copilot} to make the notes more elegent, but the correctness has not yet been verified...] Suppose that there exist scalars $c_1, c_2, \cdots, c_k$ such that \[
c_1\vt{v_1}+c_2\vt{v_2}+\cdots+c_k\vt{v_k}=\vt{0}
\]
Then we have \[
A(c_1\vt{v_1}+c_2\vt{v_2}+\cdots+c_k\vt{v_k})=c_1A\vt{v_1}+c_2A\vt{v_2}+\cdots+c_kA\vt{v_k}=\vt{0}
\]
Since $A\vt{v_i}=\lambda_i\vt{v_i}$, we have \[
c_1\lambda_1\vt{v_1}+c_2\lambda_2\vt{v_2}+\cdots+c_k\lambda_k\vt{v_k}=\vt{0}
\]
Multiplying both sides by $\vt{v_1}^T$, we get \[
c_1\lambda_1\vt{v_1}^T\vt{v_1}+c_2\lambda_2\vt{v_2}^T\vt{v_1}+\cdots+c_k\lambda_k\vt{v_k}^T\vt{v_1}=0
\]
Since $\vt{v_i}^T\vt{v_j}=0$ for $i\neq j$, we have $c_1\lambda_1\vt{v_1}^T\vt{v_1}=0$. Since $\vt{v_1}\neq \vt{0}$, we have $\vt{v_1}^T\vt{v_1}\neq 0$. Hence $c_1\lambda_1=0$. Since $\lambda_1\neq 0$, we have $c_1=0$. Similarly, we can show that $c_2=c_3=\cdots=c_k=0$. Hence the set $\{\vt{v_1}, \vt{v_2}, \cdots, \vt{v_k}\}$ is linearly independent.

\begin{example}
    Suppose that $\vt{b_1}, \vt{b_2}$ are eigenvectors corresponding to distinct eigenvalues $\lambda_1, \lambda_2$ of a matrix $A$. 
    And suppose that $\vt{b_3},\vt{b_4}$ are linearly independent vectors corresponding to a third distince eigenvalue $\lambda_3$.
    Show that the set $\{\vt{b_1}, \vt{b_2}, \vt{b_3}, \vt{b_4}\}$ is linearly independent.
\end{example}
\indent Remark: Note that under a same eigenvalue, the eigenvectors are actually the solutions to a homogeneous system of linear equations, which can be independent or dependent.

\subsection{Characteristic Equation}
\indent In the previous section, we have already tried to get the eigenvalues and eigenvectors of a matrix. 
Recall the ways we used to find the eigenvalues and eigenvectors, we can generalize the idea behind it as follows:\\
For any $n \times n$ matrix $A$, according to the definition of eigenvalues and eigenvectors, we have \[
A\vt{x}=\lambda\vt{x} \Leftrightarrow (A-\lambda I_n)\vt{x}=\vt{0}.
\]
This equation has a nontrivial solution if and only if the matrix $A-\lambda I_n$ is singular, i.e., $\det(A-\lambda I_n)=0$. 
Hence, all possible eigenvalues of $A$ are the roots of the equation $\det(A-\lambda I_n)=0$, which is called the \textbf{characteristic equation} of $A$, and the eigenvectors are the solutions to the system $(A-\lambda I_n)\vt{x}=\vt{0}$ with corresponding $\lambda$.
Furthermore, we summarize this process to get a more systematic method by giving the following definition and proposition.
\begin{definition}
    Let $A$ be an $n \times n$ matrix, then the equation \[
    \det(A-\lambda I_n)=0
    \]
    is called the \textbf{characteristic equation} of $A$.\\ 
    The polynomial \[
    p_A(\lambda)=\det(A-\lambda I_n)
    \]
    is called the \textbf{characteristic polynomial} of $A$, a polynomial of degree $n$ in variable $\lambda$.
\end{definition}
\begin{proposition}
    $\lambda$ is an eigenvalue of $A$ if and only if $\lambda$ is a root of the characteristic equation $\det(A-\lambda I_n)=0$.
\end{proposition}
\indent This proposition has connected the concept of eigenvalues with the determinant of a matrix, and consequently, the invertibility of a matrix.
Therefore, we give some new insights into the invertibility of a matrix.
\begin{quotation}
    $A:$ a $n \times n$ matrix. Then $A$ is invertible if and only if 
    \begin{itemize}
        \item The number 0 is not an eigenvalue of $A$.
        \item The determinant $\det(A)\neq 0$, and 
              \begin{equation*}
                \det A = (-1)^r \det u = \begin{cases}
                    (-1)^r\cdot(\text{product of pivots of $u$}) & \text{$u$ invertible};\\
                    0 & \text{$u$ singular},
                \end{cases}
              \end{equation*}
              where $u$ is an echelon form of $A$ by row replacements and row interchanges (without scaling) with $r$ row interchanges.
    \end{itemize}
\end{quotation}
\begin{example}
    The characteristic polynomial of a $6 \times 6$ matrix $A$ is given by \[
    \lambda^6-4\lambda^5-12\lambda^4
    \]
    Find the eigenvalues and their multiplicities.\\
    \textbf{Solution.} 
    \begin{itemize}
        \item $\lambda=0$\; multiplicity 4.
        \item $\lambda=6$\; multiplicity 1.
        \item $\lambda=-2$ multiplicity 1.
    \end{itemize}
\end{example}
\indent Remark: The purpose of putting this example here is to show the concept of \textit{multiplicity} of an eigenvalue, which is the number of times the eigenvalue appears as a root of the characteristic polynomial.
\begin{example}
    Find the eigenvalues of the matrix \[
    A = \begin{bmatrix}
        0&2&1\\
        3&0&3\\
        1&2&0
    \end{bmatrix}
    \]
    \textbf{Solution.} -1, 4, -3.
\end{example}
\subsubsection*{Similarity}
\begin{definition}
    Two $n \times n$ matrices $A$ and $B$ are called \textbf{similar} if there exists an invertible matrix $P$ such that \[
    B=P^{-1}AP\text{ or,equivalently }A=PBP^{-1}.
    \]
    Changing $A$ into $P^{-1}AP$ is called a \textbf{similarity transformation}.
\end{definition}
\begin{proposition}
    If $A$ and $B$ are similar matrices, then $A$ and $B$ have the same characteristic polynomial and hence the same eigenvalues(with the same multiplicities).
\end{proposition}
\indent \textit{Proof.} Suppose that $A$ and $B$ are similar matrices, then there exists an invertible matrix $P$ such that $B=P^{-1}AP$. 
Then we have \begin{align*}
    \det(B-\lambda I_n) &= \det(P^{-1}AP-\lambda I_n)\\
    &= \det(P^{-1}AP-\lambda P^{-1}IP)\\
    &= \det(P^{-1}(A-\lambda I_n)P)\\
    &= \det(P^{-1})\det(A-\lambda I_n)\det(P)\\
    &= \det(A-\lambda I_n).
\end{align*}
Therefore, $A$ and $B$ have the same characteristic polynomial and hence the same eigenvalues.\\
\indent \textcolor{red}{Warning:} \begin{itemize}
    \item Two matrices may have the same eigenvalues but are not similar.
    \item Similarity is not the same as row equivalence. Row operations on a matrix usually change its eigenvalues.
\end{itemize}
\subsubsection*{Application to Dynamical Systems}
\begin{example}
    Let $A$ be the matrix \[
    A = \begin{bmatrix}
        0.95 & 0.03\\
        0.05 & 0.97
    \end{bmatrix}
    \]
    Analyze the long-term behavior of the system $\vt{x}_{k+1}=A\vt{x}_k$ with $x_0=(0.6,0.4)^T$.
    \textbf{Solution.} 
    We can first get the eigenvalues of $A$ by solving the characteristic equation $\det(A-\lambda I_2)=0$, which is \[
    \lambda^2-1.92\lambda+0.92=0
    \]
    The roots are $\lambda_1=1$ and $\lambda_2=0.92$, and the corresponding eigenvectors are \[
    \vt{v_1}=\begin{bmatrix}
        3\\
        5
    \end{bmatrix}, \vt{v_2}=\begin{bmatrix}
        1\\
        -1
    \end{bmatrix}
    \]
    Then $\{v_1,v_2\}$ is obviously a basis for $\R^2$ since they are linearly independent and the dimension of $\R^2$ is 2.
    Hence, there exists weight $c_1,c_2$ such that \[
    x_0=c_1\vt{v_1}+c_2\vt{v_2} = [\vt{v_1},\vt{v_2}]\begin{bmatrix}
        c_1\\
        c_2
    \end{bmatrix}
    \]
    By multiplying both sides by $[\vt{v_1},\vt{v_2}]^{-1}$, we can get the weights $c_1,c_2 = 0.125, 0.225$.
    Since $\vt{v_1}$ and $\vt{v_2}$ are eigenvectors of $A$ corresponding to distinct eigenvalues, we can write \[
    \vt{x_1}=A\vt{x_0}=A(c_1\vt{v_1}+c_2\vt{v_2})=c_1A\vt{v_1}+c_2A\vt{v_2}=c_1\lambda_1\vt{v_1}+c_2\lambda_2\vt{v_2}
    \]
    \[
    \vt{x_2}=A\vt{x_1}=A(c_1\lambda_1\vt{v_1}+c_2\lambda_2\vt{v_2})=c_1\lambda_1^2\vt{v_1}+c_2\lambda_2^2\vt{v_2}
    \]
    \[\cdots\]
    In general, \[
    \vt{x_k}=c_1\lambda_1^k\vt{v_1}+c_2\lambda_2^k\vt{v_2}
    \]
    As $k\to \infty$, $\lambda_2^k\to 0$ since $0<\lambda_2<1$, and 
    \[
    \vt{x_k}\to c_1\vt{v_1} = 0.125\begin{bmatrix}
        3\\
        5
    \end{bmatrix}
    = \begin{bmatrix}
        0.375\\
        0.625
    \end{bmatrix}
    \]
\end{example}

\subsection{Diagonalization}

\end{document}