\documentclass[10pt, a4paper]{article}

\usepackage{listings}
\usepackage{ulem}
\usepackage{appendix}
\usepackage{amsmath, amsthm, amssymb, amsfonts}
\usepackage{thmtools}
\usepackage{graphicx}
\usepackage{setspace}
\usepackage[top=2.5cm, bottom=3.5cm, left=2.5cm, right=2.5cm]{geometry}
\usepackage{float}
\usepackage{hyperref}
\usepackage[utf8]{inputenc}
\usepackage[english]{babel}
\usepackage{ctex}
\usepackage{framed}
\usepackage[dvipsnames]{xcolor}
\usepackage{tcolorbox}
\usepackage{indentfirst}


\colorlet{LightGray}{White!90!Periwinkle}
\colorlet{LightOrange}{Orange!15}
\colorlet{LightGreen}{Green!15}

\newcommand{\HRule}[1]{\rule{\linewidth}{#1}}

\declaretheoremstyle[name=Example,]{thmsty}
\declaretheorem[style=thmsty,numberwithin=section]{example}
\tcolorboxenvironment{example}{colback=LightGray}

\declaretheoremstyle[name=Principle,]{thmsty}
\declaretheorem[style=thmsty,numberwithin=section]{principle}
\tcolorboxenvironment{principle}{colback=LightGreen}

\newenvironment{Solution}{\textbf{Solution.}}

\definecolor{codegreen}{rgb}{0,0.6,0}
\definecolor{codegray}{rgb}{0.5,0.5,0.5}
\definecolor{codepurple}{rgb}{0.58,0,0.82}
\definecolor{backcolour}{rgb}{0.95,0.95,0.92}

\lstdefinestyle{mystyle}{
    backgroundcolor=\color{backcolour},  
    commentstyle=\color{codegreen},
    keywordstyle=\color{magenta},
    numberstyle=\tiny\color{codegray},
    stringstyle=\color{codepurple},
    basicstyle=\ttfamily\footnotesize,
    breakatwhitespace=false,        
    breaklines=true,                
    captionpos=b,                    
    keepspaces=true,                
    numbers=left,                    
    numbersep=5pt,                  
    showspaces=false,                
    showstringspaces=false,
    showtabs=false,                  
    tabsize=2
}
\lstset{style=mystyle}

% ------------------------------------------------------------------------------
\setstretch{1.0}
% ------------------------------------------------------------------------------
\begin{document}
% ------------------------------------------------------------------------------
% Cover Page and ToC
% ------------------------------------------------------------------------------
\title{ \normalsize \textsc{}
\\ [2.0cm]
\HRule{1.5pt} \\
\LARGE {\textbf{\uppercase{Lecture Notes}}
\HRule{2.0pt} \\ [0.6cm] \LARGE{\textbf{PHYS1112: General Physics}} \vspace*{10\baselineskip}}
}
\date{\today}
\author{\textbf{HONG, Lanxuan}}
\maketitle
\newpage
% ------------------------------------------------------------------------------
Self-study module:\\
Analysis of motion of projectile\\
Uncertainty and significant figures\\
sliding slope model(friction)

\clearpage
% ------------------------------------------------------------------------------
% Table of Contents
% ------------------------------------------------------------------------------
\tableofcontents
\newpage
% ------------------------------------------------------------------------------
% Here is the main section
% ------------------------------------------------------------------------------
\section{MECHANICS}
\subsection{Units, Physical Quantities, and Vectors}
As an experimental science, physics deals with measured quantities. We use numbers for the measurements and when doing so, we compare the observed quantities with some gauge.\footnote{pre-defined measures}\\

SI system of units are commonly used in physics. Basic units includes kilogram\footnote{It is defined by comparing to the mass of a fixed "etalon" object}, meter\footnote{Lights travels exactly 299,792,485 meters in 1 second. This shows that meter is measured after precise measurement of second.} and second\footnote{Scientists use the microwave radiation with a 9,192,631,770 cycles per second of cesium-133 atom to measure the second}.\\

Vector calculation is very important in physics. The note skip this part and please refer to MATH 1014 lecture note.
\subsection{Frame of Reference \& Relative Velocity}
Consider the case that an observer making a measurement from his view. This form the frame of reference. Based on whether they follow the Newton's motion law(second law), the frame of reference have 2 types: the inertial frame of reference and the non-inertial frame of reference. \textbf{The Inertial frame of reference} is the frame of reference in which the Newton's motion law works. By contrary, the Newton's motion law does not work in a non-inertial frame of reference. Here is an example of observing the same object from different aspect which form different frame of reference
\begin{example}
\textbf{Observing objects and frame of reference}
    \begin{enumerate}
        \item A passer-by observing a passenger sitting in a moving train.\footnote{Earth is not an inertial frame of reference. However, in most of discussion, we can say that the surface of Earth is approximately an inertial reference frame. }
        \item Another passenger on the train observing this passenger sitting in the train.
    \end{enumerate}
\end{example}
    Any reference frame, which moves with constant velocity with respect to an inertial frame is also an inertial frame.\footnote{Do vector calculation and notice that the additional vector is constant} In other word, we can consider this frame as a frame without acceleration.\\

As for the \textbf{non-inertial frame of reference}, when observing from inertial frame of reference, it has certain acceleration. As a result, one may feel like being acted on by a force in a non-inertial frame of reference.\footnote{More information about non-inertial frame of reference.\href{https://zhuanlan.zhihu.com/p/62150288}{\uline{\textit{Fictitious Forces Here}}}}  One important usage of fictitious forces is using rotation to provide a sense of gravity in the space.
\begin{example}
    \textbf{Analysis of the motion of the projectile}
\end{example}
\clearpage
\subsection{Newton's Law}
\begin{principle}
    \textbf{Newton's Law of Motion}
    \begin{itemize}
        \item Newton's First Law: A body acted on by no net force moves with constant velocity.\footnote{constant speed and straight line}
        \item Newton's Second Law: If a net external force acts on a body, the body accelerates according to the following relation:
    $$
    \vec{F} = m\sum \vec{a}
    $$
        \item Newton's Third Law: Body A exerts a force on body B\footnote{which is called "Action"}, then the Body B exerts a force on body A.\footnote{which is called "Reaction"}
    \end{itemize}
\end{principle}
In real life experience, it is hard to find an object with no force acting on. So Newton's first law also apply to bodies in equilibrium.\footnote{sum of external forces is 0} It is important to note that Newton's First and Second Law can be used only in inertial frame of reference, while Newton's Third Law can be used in both frame of reference.\\

\end{document}