\documentclass[a4paper,12pt]{article}
\usepackage{amsmath}
\usepackage{amsfonts}
\usepackage{amssymb}
\usepackage{geometry}
\usepackage{multicol}
\geometry{left=0.3in}
\geometry{right=0.3in}
\geometry{top=0.3in}
\geometry{bottom=0.5in}
\setlength{\parindent}{2em}
\begin{document}

\begin{multicols}{2}
\setlength{\columnseprule}{0.4pt}

\subsection*{- Basic Distributions -}
\subsubsection*{Binomial Distribution $X\sim Bin(n,p)$}
If x is the number of successes in n independent Bernoulli trials, ($Be(p)=Bin(1,p)$)
\[
P(X = x) = \binom{n}{x}p^x(1-p)^{n-x}
\]
$E(X) = np, \; Var(X) = np(1-p), \; M_X(t) = (1-p+pe^t)^n$

\subsubsection*{Geometric Distribution $X\sim Geo(p)$}
Define the random variable $X$ as the number of trials until the first success. Then the probability mass function of $X$ is:
\[
P(X = x) = (1-p)^{x-1}p
\]
$E(X) = \frac{1}{p}, \; Var(X) = \frac{1-p}{p^2}, \; M_X(t) = \frac{pe^t}{1-(1-p)e^t}$

\subsubsection*{Negative Binomial $X\sim NB(r,p)$}
Define the random variable $X$ as the number of trials until the $r$th success. Then the probability mass function of $X$ is:
\[
P(X = x) = \binom{x-1}{r-1}p^r(1-p)^{x-r}
\]
$E(X) = \frac{r}{p}, \; Var(X) = \frac{r(1-p)}{p^2}$

\subsubsection*{Hypergeometric $X\sim HGeom(N,K,n)$}
If $X$ is the number of successes in $n$ draws without replacement from a population of $N$ objects of which $K$ are successes, then the probability mass function of $X$ is:
\[
P(X = x) = \frac{\binom{K}{x}\binom{N-K}{n-x}}{\binom{N}{n}}
\]
$E(X) = \frac{nK}{N}, \; Var(X) = \frac{nK(N-K)(N-n)}{N^2(N-1)}$

\subsubsection*{Poisson $X\sim Poisson(\lambda)$}
If $X$ is the number of events in a fixed interval of time or space ($\lambda$: rate of occurence per unit time), then the probability mass function of $X$ is:
\[
P(X = x) = \frac{e^{-\lambda}\lambda^x}{x!}
\]
$$E(X) = \lambda, \;Var(X) = \lambda, \; M_X(t) = e^{\lambda(e^t-1)}$$
Poison random variables can be used as approximations for binomial random variables when $n$ is large and $p$ is small enough so that $\lambda = np$ is moderate(usually if $n>20$ and $np<15$). 

\subsubsection*{Uniform $X\sim U(a,b)$}
\[
f_X(x) = \begin{cases}
\frac{1}{b-a} & \text{if } a\leq x\leq b \\
0 & \text{otherwise}
\end{cases}
\]
$E(X) = \frac{a+b}{2}, \; Var(X) = \frac{(b-a)^2}{12}, \; M_X(t) = \frac{e^{tb}-e^{ta}}{t(b-a)}$

\subsubsection*{Normal $X\sim N(\mu,\sigma^2)$}
\[
f_X(x) = \frac{1}{\sqrt{2\pi}\sigma}e^{-\frac{(x-\mu)^2}{2\sigma^2}}, \; -\infty < x < \infty
\]
For standard normal distribution, $Z\sim N(0,1)$, we denote its density function $\phi(z)$ and its distribution function $\Phi(z)$. 
Remember it's imposible to integrate the normal distribution function except the $-\infty \to \infty$ case.
The $q$-th quantile of of a random variable $X$ is defined as a number $z_q$ such that $P(X<z_q)=q$.

\subsubsection*{Gamma Distribution}
Gamma function is $\Gamma(\alpha)=\int_{0}^{\infty}x^{\alpha-1}e^{-x}dx$, and we say $X$ follows Gamma distribution($X\sim \Gamma(\alpha, \lambda)$) if 
\[
f_X(x) = \begin{cases}
\frac{\lambda^{\alpha}x^{\alpha-1}e^{-\lambda x}}{\Gamma(\alpha)} & \text{if } x>0 \\
0 & \text{otherwise}
\end{cases}
\]
where $\alpha,\,\lambda>0$, and $E(X) = \frac{\alpha}{\lambda}$, $Var(X) = \frac{\alpha}{\lambda^2}$.\\
\indent When $\alpha=1$, $\Gamma(1,\lambda)=\text{Exp}(\lambda)$, which has the unique property of memorylessness. 
The Gamma distrubution can also be thought of as a waiting time between Poisson distributed events.

\subsubsection*{Beta Distribution}
Beta function is $B(a,b)=\int_0^1x^{a-1}(1-x)^{b-1}dx$, and we say $X$ follows Beta distribution($X\sim \text{Beta}(a,b)$) if
\[
f_X(x) = \begin{cases}
\frac{x^{a-1}(1-x)^{b-1}}{B(a,b)} & \text{if } 0<x<1 \\
0 & \text{otherwise}
\end{cases}
\]
where $a,b>0$, and $E(X) = \frac{a}{a+b}$, $Var(X) = \frac{ab}{(a+b)^2}$, by showing that $B(a,b)=\frac{\Gamma(a)\Gamma(b)}{\Gamma(a+b)}$.

\subsubsection*{Cauchy $X\sim Cauchy(\theta)$}
\[
f_X(x) = \frac{1}{\pi}\frac{1}{1+(x-\theta)^2}, \; -\infty < x < \infty
\]
Cauchy distribution has no mean or variance (both diverges) although it is a valid pdf, and its median is $\theta$. The ratio of two N(0,1) is Cauchy(0).

\subsection*{- Calculation Tricks -}
\(
f(x) = \sum_{k=0}^{\infty} \frac{f^{(k)}(0)}{k!}x^k = f(0) + f'(0)x + \frac{f''(0)}{2!}x^2 + \cdots
\)\\
\(
\int x e^{ax}dx = \frac{1}{a^2}e^{ax}(ax-1)
\)\\
\(
\int x^2 e^{ax}dx = \frac{1}{a^3}e^{ax}(a^2x^2-2ax+2)
\)\\
\(
\int x^3 e^{ax}dx = \frac{1}{a^4}e^{ax}(a^3x^3-3a^2x^2+6ax-6)
\)

\newpage

\subsection*{- Transformations of RV -}
\subsubsection*{Real-valued Functions of RV}
For discrete random variables $X$ and $Y$:
\[
p_{g(X,Y)}(z) = \sum_{x,y:g(x,y)=z}p_{X,Y}(x,y)
\]
For continuous random variables $X$ and $Y$:
\[
f_{g(X,Y)}(z) = \frac{d}{dz} \iint\limits_{g(x,y)=z} f_{X,Y}(x,y) \, dx \, dy
\]
By calculations, we have these special cases:
\begin{align*}
    f_{X+Y}(z) &= \int_{-\infty}^{\infty}f_{X,Y}(x,z-x)dx\\
    f_{X-Y}(z) &= \int_{-\infty}^{\infty}f_{X,Y}(x,x-z)dx\\
    f_{XY}(z) &= \int_{-\infty}^{\infty}\frac{1}{|x|}f_{X,Y}(x,\frac{z}{x})dx\\
    f_{X/Y}(z) &= \int_{-\infty}^{\infty}|y|f_{X,Y}(zy,y)dy
\end{align*}

\subsubsection*{Vector-valued F of RV (Change of Variables)} 
Single integral:
\[
\int_{x=a}^{x=b} f(x)dx = \int_{y=g(a)}^{y=g(b)} f(g^{-1}(y))\frac{dx}{dy}dy
\]
Using the property of monotonicity, when $Y = g(X)$:
\[
f_Y(y) = f_X(g^{-1}(y))|\frac{dx}{dy}|
\]
Double integral:
\[
\iint\limits_R f_(x,y) \, dx \, dy = \iint\limits_{S} f_(u,v) \left|\frac{\partial(x,y)}{\partial(u,v)}\right| \, du \, dv
\]
where $S$ is the region in the $uv$ plane that corresponds to region $R$ in the $xy$ plane, and 
\begin{align*}
\left|\frac{\partial(x,y)}{\partial(u,v)}\right| 
&= \begin{vmatrix}
\frac{\partial x}{\partial u} & \frac{\partial x}{\partial v} \\
\frac{\partial y}{\partial u} & \frac{\partial y}{\partial v}
\end{vmatrix} =
\left|\frac{\partial x}{\partial u}\frac{\partial y}{\partial v} - \frac{\partial x}{\partial v}\frac{\partial y}{\partial u}\right|\\
&= \begin{vmatrix}
\frac{\partial u}{\partial x} & \frac{\partial u}{\partial y} \\
\frac{\partial v}{\partial x} & \frac{\partial v}{\partial y}
\end{vmatrix}^{-1} =
\left|\frac{\partial u}{\partial x}\frac{\partial v}{\partial y} - \frac{\partial u}{\partial y}\frac{\partial v}{\partial x}\right|^{-1}
\end{align*}
Hence, we have
\[
f_{U,V}(u,v) = f_{X,Y}(x(u,v),y(u,v))\left|\frac{\partial(x,y)}{\partial(u,v)}\right|
\]

\subsection*{- Integration Techniques -}
By parts:
\(
\int u dv = uv - \int v du
\)\\
Substitution:
\(
\int f(g(x))g'(x)dx = \int f(u)du
\)


\newcolumn

\subsection*{- Expectation and Variance -}
\subsubsection*{Expectation}
\[
E[g(X,Y)] = \begin{cases}
\sum_{x,y}g(x,y)p_{X,Y}(x,y) & \text{dis}\sim \\
\int_{-\infty}^{\infty}\int_{-\infty}^{\infty}g(x,y)f_{X,Y}(x,y) \, dx \, dy & \text{con}\sim
\end{cases}
\]
$$E(X) = \sum_{x}P(X>x) \text{ or } \int_{0}^{\infty}P(X>x)dx$$
Linearity property (application: Boole's inequality):
\[
E\left[\sum_{i=1}^{n}a_iX_i\right] = \sum_{i=1}^{n}a_iE[X_i]
\]
Independence property (converse not true):
\[
E[g(X)h(Y)] = E[g(X)]E[h(Y)]
\]
eg. Coupon-collecting: one type one RV
\subsubsection*{Covariance and Correlation}
\[\text{cov(X,Y)} = E(X-\mu_X) (Y-\mu_Y) = E(XY) - \mu_X\mu_Y\]\[
 \rho (X,Y) = \frac{\text{cov(X,Y)}}{\sigma_X\sigma_Y}\in[-1,1]\]
Linearity property (No.3 - Independence):
\[
\text{cov}(aX+b,cY+d) = ac\cdot\text{cov}(X,Y)
\]
\[
\text{cov}(\sum_{i=1}^{n}a_iX_i,\sum_{j=1}^{m}b_jY_j) = \sum_{i=1}^{n}\sum_{j=1}^{m}a_ib_j\text{cov}(X_i,Y_j)
\]
\[
\text{var}(\sum_{k=1}^{n} X_k) = \sum_{k=1}^{n}\text{var}(X_k) + 2\sum_{i=1}^{n-1}\sum_{j=i+1}^{n}\text{cov}(X_i,X_j)
\]

\subsubsection*{Conditional Expectation and Variance}
Expectation:
\[
E[X|Y=y]:=\int_{-\infty}^{\infty}xf_{X|Y}(x|y)dx = \int_{-\infty}^{\infty}x\frac{f_{X,Y}(x,y)}{f_Y(y)}dx
\]
\[
E[X] = E[E[X|Y]] = \int_{-\infty}^{\infty}E[X|Y=y]f_Y(y)dy
\]
Variance:
\[
Var(X|Y=y) = E[(X-E[X|Y=y])^2|Y=y]
\]
\[
var(X) = E[var(X|Y)] + var(E[X|Y])
\]

\subsection*{- Moment Generating Function -}
\[
M_X (t) = E[e^{tX}] = \int_{-\infty}^{\infty} e^{tx}f_X(x)dx
\]
\[
E(X^n) = M_X^{(n)}(0) \frac{d^n}{dt^n}M_X(t)|_{t=0}
\]
\[
M_{X,Y}(s,t) = E[e^{sX+tY}] = \int_{-\infty}^{\infty}\int_{-\infty}^{\infty} e^{sx+ty}f_{X,Y}(x,y)dxdy
\]
\indent Can be used to determine distribution and independence!

\newpage

\end{multicols}

\end{document}