\documentclass[a4paper,12pt]{article}
\usepackage{amsmath}
\usepackage{amsfonts}
\usepackage{amssymb}
\usepackage{geometry}
\usepackage{multicol}
\geometry{margin=1in}
\setlength{\parindent}{2em}
\begin{document}

\begin{multicols}{2}
\section*{- Combinatorial Analysis -}

\setlength{\columnseprule}{0.4pt}
\subsubsection*{Distinct Objects, Distinct Boxes}
The number of permutations of $n$ objects of which $n_1$ are of one kind, $n_2$ are of a second kind, $\ldots$, and $n_k$ are of a $k$th kind is
\[
\frac{n!}{n_1!n_2!\cdots n_k!} = \binom{n}{n_1,n_2,\ldots,n_k}
\]

\subsubsection*{Identical Objects, Distinct Boxes}
The number of ways of distributing $n$ identical objects into $r$ distinct boxes is the solution to the equation
\[
x_1 + x_2 + \cdots + x_r = n
\]
where $x_i$ is the number of objects in the $i$th box. The number of solutions is
\[
\binom{n+r-1}{r-1}
\]
When we require that each box contain at least one object, the number of solutions is
\[
\binom{n-1}{r-1}
\]
The proof can also be achieved by using the stars and bars method.

\subsubsection*{Distinct Objects, Identical Boxes}
Whenever there are two or more identical boxes, remember to divide by the number of ways the boxes can be permuted. The number of ways of distributing $n$ distinct objects into $r$ identical boxes is
\[
\binom{n+r-1}{r-1}\frac{1}{r!}
\]

\subsubsection*{Multinomial Coefficients}
The expansion of $(x_1 + x_2 + \cdots + x_k)^n$ is
\[
\sum_{n_1 + n_2 + \cdots + n_k = n} \frac{n!}{n_1!n_2!\cdots n_k!}x_1^{n_1}x_2^{n_2}\cdots x_k^{n_k}
\]
\newcolumn

\section*{- Probability -}
\subsubsection*{Concepts}
\begin{itemize}
    \item primitive defination of probability: limiting frequency
    \item Mutual Exclusivity: pairwise disjoint
    \item Exhaustivity: union of all events is the sample space
\end{itemize}
\subsubsection*{Probability Axioms}
\begin{enumerate}
    \item $P(S) = 1$
    \item $P(E)\geq 0$
    \item mutually exclusive, straight addition
\end{enumerate}
\subsubsection*{Inclusion-Exclusion Principle}
\[
P(\bigcup_{i=1}^n E_i) = \sum_{i=1}^{n}(-1)^{i+1}[\sum_{1<j_1<\cdots<j_i\leq n}P(\bigcap_{k=1}^i E_{j_k})]
\]
\subsubsection*{Bayes' Theorem}
\[
P(B_j|A) = \frac{P(A|B_j)P(B_j)}{\sum_{i=1}^{n}P(A|B_i)P(B_i)}
\]
\begin{itemize}
    \item sensitivity = probability of a positive test given that the patient has the disease $=\frac{TP}{TP+FN}$
    \item specitivity = probability of a negative test given that the patient does not have the disease $=\frac{TN}{TN+FP}$
    \item prevalence = probability of having the disease $=\frac{TP+FN}{TP+FN+TN+FP}$
\end{itemize}

\newpage
\section*{- Random Variables -} 
Over a sample space $S$, a random variable $X$ is a function with the map: 
\[
X: S \rightarrow \mathbb{R}
\]
or, if $X$ is discrete, the map:
\[
X: S \rightarrow \mathbb{Z}
\]
For any proofs invoving random variables, remember to use the definition of the random variable and probability of random variables.

\subsubsection*{Bernoulli Distribution}
\[
X = \begin{cases}
1 & \text{with probability } p \\
0 & \text{with probability } 1-p
\end{cases}
\]
We say $X\sim Be(p)$, and $$E(X) = p, \; Var(X) = p(1-p)$$

\subsubsection*{Binomial Distribution}
If x is the number of successes in n independent Bernoulli trials, then the probability mass function of x is:
\[
P(X = x) = \binom{n}{x}p^x(1-p)^{n-x}
\]
We say $X\sim Bin(n,p)$, and $$E(X) = np, \; Var(X) = np(1-p)$$

\subsubsection*{Geometric Distribution}
Define the random variable $X$ as the number of trials until the first success. Then the probability mass function of $X$ is:
\[
P(X = x) = (1-p)^{x-1}p
\]
We say $X\sim Geo(p)$, and $$E(X) = \frac{1}{p}, \; Var(X) = \frac{1-p}{p^2}$$

\subsubsection*{negative binomial distribution}
Define the random variable $X$ as the number of trials until the $r$th success. Then the probability mass function of $X$ is:
\[
P(X = x) = \binom{x-1}{r-1}p^r(1-p)^{x-r}
\]
We say $X\sim NB(r,p)$, and $$E(X) = \frac{r}{p}, \; Var(X) = \frac{r(1-p)}{p^2}$$

\subsubsection*{Hypergeometric Distribution}
If $X$ is the number of successes in $n$ draws without replacement from a population of $N$ objects of which $K$ are successes, then the probability mass function of $X$ is:
\[
P(X = x) = \frac{\binom{K}{x}\binom{N-K}{n-x}}{\binom{N}{n}}
\]
We say $X\sim HGeom(N,K,n)$, and $$E(X) = \frac{nK}{N}, \; Var(X) = \frac{nK(N-K)(N-n)}{N^2(N-1)}$$

\subsubsection*{Poisson Distribution}
If $X$ is the number of events in a fixed interval of time or space, then the probability mass function of $X$ is:
\[
P(X = x) = \frac{e^{-\lambda}\lambda^x}{x!}
\]
We say $X\sim Pois(\lambda)$, and $$E(X) = \lambda, \;Var(X) = \lambda$$
Poison random variables can be used as approximations for binomial random variables when $n$ is large and $p$ is small enough so that $\lambda = np$ is moderate(usually if $n>20$ and $np<15$). 



\newpage


\end{multicols}

\end{document}